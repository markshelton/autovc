\documentclass[../thesis/thesis.tex]{subfiles}
\begin{document}
 \chapter{Evaluation}

We believe it is possible to produce a VC investment screening system that is efficient, robust and powerful. In the previous chapter, we describe the development and structure of such a system. Our system is based around identifying startup companies that are likely to receive additional funding or a liquidity event (exit) in a given forecast window. This system can generate statistics and make recommendations that may assist VC firms to efficiently and effectively screen investment candidates. In our literature review, we determined that our proposed system must satisfy three criteria to be useful to VC firms: efficiency, robustness and predictive power. In this chapter, we evaluate our system against that criteria.

Firstly, we evaluate efficiency by exploring the learning curves of our classification techniques and whether there is sufficient data to produce reliable statistics. We also evaluate efficiency with respect to accuracy at different levels of feature abstraction (e.g. feature grouping based on our conceptual framework). Secondly, we evaluate robustness by evaluating our models against multiple reverse-engineered historical datasets and measuring their variance. Thirdly, we evaluate predictive power by testing different forecast windows outcomes and evaluating our models’ accuracy for startups at different stages of their development lifecycle. Finally, we discuss our findings more broadly and their implications for investors and future research into startup investment and performance.

\section{Efficiency}

A key rationale for this research project is the need for more efficient forms of VC investment analysis which leverage technology, particularly in surfacing and screening. By its nature, our automated system should be more efficient than current manual methods but it's also important to determine how to most efficiently use the data that we have. For these reasons, we decided to investigate the learning curves for our classification pipeline to determine whether we could use smaller samples from our dataset to achieve similar predictive power and reduce our need for computational power and time taken. We used cross-validation to split our dataset 10 times and used subsets of the training set with varying sizes to train the estimator and produce training and test scores for each subset size. The convergence or divergence of our training and cross-validation curves will imply whether our classification pipeline is over- or under-fitting our data for various sizes allowing us to select an optimal sample size. Figure 4.1 shows our learning curves

\section{Robustness}

It is critical that our system be robust with respect to time so investors can rely on its future predictions based on historical models. CrunchBase provides created and last-updated timestamps for each record in their CSV-formatted dumps (and also in the JSON-formatted responses from their API). We took advantage of this to produce a system that  reverse-engineers previous database states by filtering the current database by only records that were created by a given ‘slice’ date. We performed preliminary testing of this technique by comparing a historical CrunchBase database collected in Dec-2013 with a slice from our primary dataset collected in Sep-2016, as shown in Table 3.5. While there are some differences in the records, particularly in the IPO relation, we consider this to be satisfactory variance considering the 3-year time period. The key relations for our purposes are Companies, Funding Rounds and People, all of which had minor differences considering the size of these datasets. Table 3.6 shows company counts by startup development stage from different dataset slices. We limited our experiments to dataset slices from 2012-onwards because prior to 2012 the datasets become too small to use to make predictions (given the class imbalance).

\section{Predictive Power}

Finally, we test our system’s predictive power for making forecasts of different time periods. We use the same system of reverse-engineering time slices that we used in our previous experiment to robustness, but this time we vary the time difference between the slice that provides our features and the slice that provides our label. Preliminary testing on a Apr-2012 features slice collected from the Sep-2016 is shown in Table 3.7. Very few companies appear to change stage over a period of less than 2 years so we will focus our experimentation on forecast windows of 2-4 years. We performed testing against labels from the Apr-2017 dataset.

 \ifcsdef{mainfile}{}{\printbibliography}
\end{document}
