\documentclass[../thesis/thesis.tex]{subfiles}
\begin{document}

 \chapter{Conclusions}
 \label{chap:conclusions}

Our project's aim was to produce an investment screening system suitable for use in the \gls{vc} industry. Our system identifies startups that are likely to raise additional funding, become acquired or have an \gls{ipo} (or some combination thereof) in a given period. While this is a challenging task, our system achieved results that have practical application for \gls{vc} firms.

\section{Evaluation of Criteria}

We evaluated our system against three criteria: practicality, robustness, and versatility.

\subsection{Practicality}

Investment screening involves considerable time and effort for \gls{vc} firms~\cite{fried1994}. We evaluated whether it is practical for our system to replace existing screening processes (e.g. Google search, industry papers, databases). Our automated system is more efficient than these methods because it requires minimal user input. Our system takes 46 hours to run, which is reasonable because screening is not time-sensitive in this industry. The majority of the time taken by our system is due to pipeline optimisation. In the future, we could develop a scheduling system that runs time-intensive components (like pipeline optimisation) less frequently with minimal reduction in performance.

\subsection{Robustness}

\Gls{vc} firms are concerned that investment models trained on historical data will not accurately predict future trends and activity~\cite{stone2014}. We evaluated whether our system's predictions are robust over time. We found that training our system on different historical datasets had a minimal effect on the system's performance and the models it generated. This finding suggests that \gls{vc} firms should be able to act on predictions made by our system. Our system is also robust to dataset size because it chooses an optimal classification pipeline based on the dataset available. It is likely that our system will continue to improve in performance as its data sources grow.

\subsection{Versatility}

\Gls{vc} firms vary in the investments they make according to their interests, the lifecycles of their funds, and the portfolios that they hold~\cite{gompers1995}. A \gls{vc} investment screening system should be versatile in its ability to work across these variables. We evaluated our system's ability to perform across a large domain of investment prediction tasks. Tasks included predicting different target outcomes (e.g. \gls{ipo}, acquisition) for companies at different developmental stages (e.g. Seed, Series A, etc.) over different forecast windows (e.g. 2--4 years). These variables have significant effects on our system's performance and the models it generates. Where comparable, our system produced better or similar results to previous attempts.

\section{Future Work}

We identified several areas of further investigation that build on our work.

\subsection{Systems Integration}

Our system was designed to meet criteria critical to \gls{vc} firms, but we must now assess how to integrate our software into their systems. First, we hope to conduct a use case study with one or more \gls{vc} firms. This study will inform the commercialisation of our system. One potential area of further development is an autonomous system that schedules the system components to ensure consistent near-optimal performance. This task scheduling system would pair well with an improved CrunchBase data collection system that connects to the CrunchBase API rather than downloading CSV-dumps. An API connector would allow the task scheduler to run an optimisation process that maximises performance improvement against dataset changes and time taken.

\subsection{Feature Improvement}

This project provided a comprehensive study of features that predict startup performance, but there remain improvements we can make to the feature set. Some factors from our framework were not represented or supported by few features: media coverage, strategic alliances, financial performance, social influence, and industry performance. We expect their inclusion to improve our predictions. We also expect that more complex features like semantic text analysis (e.g. keyword analysis from patents, sentiment analysis from media) and social network analysis (e.g. co-investment networks, spheres of influence) will improve our predictions. Finally, we hope to explore whether temporal relationships between startup activities (e.g. media coverage, funding rounds, \gls{ipo}, etc.) might also improve performance.

\section{Summary}

We set out to create a \gls{vc} investment screening system that met criteria of practicality, robustness, and versatility, and we have indeed created such a system. The work required to achieve this project's goals was extensive, from reviewing the state of \gls{vc} theory and data mining, to developing systems that collect data from CrunchBase and PatentsView, to developing an adaptive classification pipeline process, and finally performing experiments that validate the ability of the system to meet our criteria above.

This project makes three primary contributions with implications for industry and research. First, we designed our system for the \gls{vc} industry: it is near-autonomous, robust to changes in dataset and prediction task, and uses a comprehensive feature set collected from large public online databases. Second, our system's performance is not only better or comparable to previous studies, but it also addresses a far larger domain of investment prediction tasks with respect to forecast window, developmental stage and target outcome. Third, this project contributes an empirical study of models of startup investment performance more comprehensive than any found in the literature. Ultimately, this project makes steps towards automation in the \gls{vc} industry.

 \ifcsdef{mainfile}{}{\printbibliography}
\end{document}
