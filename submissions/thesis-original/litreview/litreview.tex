\documentclass[../thesis/thesis.tex]{subfiles}
\begin{document}

\chapter{Literature Review}
\label{chap:litreview}

In this chapter, we first review the state of \gls{vc} investment and develop criteria which we can use to evaluate \gls{vc} investment screening systems. We then determine the best methodologies to adopt to develop such a system, which we break down into three areas: features, data sources and classification algorithms.

\begin{enumerate}

\item Business Context. \Gls{vc} firms perform screening on many potential investment candidates. Current screening methods are time-consuming and subject to human selection biases. Preliminary evidence suggests that \gls{vc} firms would be likely to adopt an improved investment screening system that is practical, robust and versatile.

\item Features. \Gls{vc} investment is a key driver of startup development yet we have an incomplete understanding of the factors that influence \gls{vc} investment decisions and the subsequent performance of those investments. Previous studies have explored a range of features, the most significant of which are human capital, economic conditions, and investment record. However, few individual studies have evaluated a comprehensive and diverse feature set.

\item Data Sources. Startup performance is a multi-faceted problem and different data sources provide different perspectives on its actors, relationships and attributes. We review online data sources which have the potential to provide large, diverse feature sets. Preliminary evidence suggests startup databases CrunchBase and AngelList are promising sources.

\item Classification Algorithms. Predicting startup performance is difficult. However, machine learning has been applied successfully in other areas of finance. We evaluate common classification algorithms with respect to their suitability for our problem and intended feature set. We conclude that Random Forests, Support Vector Machines and Artificial Neural Networks appear to be most suitable for \gls{vc} investment screening systems.

\end{enumerate}

\section{Business Context}

In this section, we provide an introduction to the business context of \gls{vc} investment screening. We review \gls{vc} firm strategy and their investment processes. We explore why the adoption of data mining in the \gls{vc} industry has been slow and develop criteria by which we can evaluate whether a \gls{vc} investment screening system is an improvement on current methods.

\subsection{Venture Capital Industry}

\Gls{vc} investment is a key enabler of technological innovation and critical to research and technology-intensive industries (e.g software, medical and agricultural technologies). \Gls{vc} investment is a form of private equity, a medium to long-term form of finance provided in return for an equity stake in potentially high-growth companies. \gls{vc} investment, in comparison to other forms of finance, is characterised by a large number of investment candidates, high degree of uncertainty, a lack of reliable data on company performance (particularly financial performance), and extensive due-diligence.

\Gls{vc} firms are reliant on a small number of high-risk investments to produce outsized returns through successful exit events. A rule-of-thumb is that given a \gls{vc} portfolio of ten investments: three will fail entirely, three will remain active but not be very profitable, three will be active and profitable, and one will be highly successful and provide the firm with a return on all of their investments~\cite{stone2014}. Compared to other forms of finance, \gls{vc} investment is biased towards control at the expense of risk mitigation. Although \gls{vc} firms tend not to take majority stakes, they exert influence through significant minority stakes, board membership, and leveraging their networks~\cite{fried2006}.

\subsection{Venture Capital Systems}

The \gls{vc} investment process involves several main stages: investment origination and screening, evaluation, structuring (e.g., valuation, term sheets), and post-investment activities (e.g., recruiting, financing). In this project, we are most interested in the first stage: origination and screening (which we refer to as `screening'). This process is complicated. As the cost of starting businesses has decreased, investors are faced with an increasing number of investment opportunities to evaluate~\cite{graham2013}. At the same, despite \gls{vc} firms' considerable influence on the trajectory of their investments, they remain highly selective. Studies show \gls{vc} investment rates vary between 1.5--3.5\% of proposals considered~\cite{stone2014}.

Referrals from trusted sources (e.g. portfolio entrepreneurs, other investors) are often used by \gls{vc} firm to initially screen opportunities. Some \gls{vc} firms are known to ignore approaches that do not have a qualified referral. \Gls{vc} firms may also discover companies through papers, events and databases related to the startup scene or particular industries. These screening processes are labour-intensive and time-consuming, and even after a company is screened more due-diligence is needed before a final decision is made.

\subsection{Evaluation Criteria}

Despite evidence that \gls{vc} firms might benefit from data mining, the \gls{vc} industry has lagged behind other forms of finance (e.g. bond trading, loan applications, insurance) in adopting technology to aid their decision-making. Banks are able to evaluate loan requests in minutes while \gls{vc} firms take far longer to put together deals, sometimes months. While these are markedly different forms of finance (\gls{vc} has a longer return period, larger investments, higher risk profiles), a more data-informed and analytical approach to \gls{vc} investment is foreseeable.

The \gls{vc} industry's view on applying data mining techniques to investment screening can be broadly characterised as interested but cautious. \Gls{vc} firm adoption of databases like CrunchBase and AngelList has increased in recent years~\cite{patil2015}. However, firms do not appear to be using these databases to build automated systems. Stone (2014) interviewed Fred Wilson of Union Square Ventures who said: ``We have not been able to quantify [startup potential]. We haven't even tried. Although I am sure someone could do it and they might be very successful with it. To us, the ideal founding team is one supremely talented product oriented founder and one, two, or three strong developers, and nothing else.''~\cite{stone2014}.

Based on our review of the \gls{vc} industry and current \gls{vc} screening processes, we have developed criteria by which we can evaluate \gls{vc} investment screening systems:

\begin{enumerate}

\item Practicality. An improved \gls{vc} investment screening system must be more practical than manual investment screening (e.g. referrals, research papers, Google search). If the system is not easy to use then it is unlikely to be adopted because of inertia, the cost of re-training staff, and limited technical expertise in the industry. Most \gls{vc} firms are small in headcount so these effects may have more impact than in other forms of finance (e.g. banking). An ideal system might be one that runs in the background and makes recommendations on the front-end. This could strike a good balance between being helpful while not taking away control from \gls{vc} firm~\cite{fried2006}. Investment screening is not considered a time-sensitive process in the \gls{vc} industry (unlike structuring, for example), but screening systems should be designed to process new data quickly enough that the predictions are up-to-date and relevant.

\item Robustness. An improved \gls{vc} investment screening system must be robust to changes over time. \Gls{vc} firms have concerns over the quality and volatility of factors used in data mining to predict startup performance. Chris Dixon of Andreessen Horowitz stated: ``I've seen a few attempts to do it quantitatively but I think those are often flawed because the quantitatively measurable things are either obvious, irrelevant, or suffer from over-fitting.''~\cite{stone2014}. Therefore, a screening system should have minimal variance in performance when training on datasets from different dates so investors can trust its ability to make future-looking predictions. In addition to robust performance, \gls{vc} firms seek systems that are themselves robust to time and will not become quickly outdated. Systems should be able to adapt to new data sources and feature sets as they become available.

\item Versatility. An improved \gls{vc} investment screening system must be able to address a large domain of investment prediction tasks. \Gls{vc} firms vary in the investments they make according to their interests, the life-cycles of their funds, and the portfolios that they hold~\cite{gompers1995}. For example, \gls{vc} firms make investment decisions with different periods so they can strategically manage the investment horizons of their funds. An investment screening system should be versatile enough to to make accurate predictions for companies of different developmental stages (e.g. Seed, Series A), for different target outcomes (e.g. Acquisition, IPO) across different forecast windows (e.g. exit in two years, exit in four years).

\end{enumerate}

\section{Features}

In this section, we provide a review of studies performed into \gls{vc} investment decisions and startup performance. In Table~\ref{tab:litreview:features:summary} we evaluate factors that have been indicated to influence \gls{vc} investment decisions and might be relevant to \gls{vc} investment screening systems.

\begin{table}[!htb]
    \centering
    \scalebox{1}{
        
\newcommand{\factor}[1]{\hspace{-4em}#1}
\newcommand{\group}[1]{\hspace{-2em}#1}

\begin{tabular}{>{\hspace{4em}}lll}
\toprule
\multicolumn{1}{l}{Features} & \multicolumn{2}{c}{Results from Studies} \\
\cmidrule(lr){2-3}
 & Significant & Non-Significant \\
\midrule
\factor{Startup Potential} \\
      \group{Human Capital} \\
            Founder Capabilities
                  & \cite{beckwith2016,an2015,gimmon2010}
                  & \cite{shan2014,conti2013} \\
            Advisor Capabilities
                  & \cite{baum2004}
                  & \cite{ahlers2015,an2015} \\
            Executive Capabilities
                  & \cite{beckwith2016,an2015,conti2013}
                  & \cite{ahlers2015} \\
      \group{Social Capital} \\
            Strategic Alliances
                  & \cite{baum2004}
                  & - \\
            Social Influence
                  & \cite{beckwith2016,an2015,cheng2016,yu2015}
                  & - \\
      \group{Structural Capital} \\
            Patent Filings
                  & \cite{hoenen2014,hsu2008,baum2004}
                  & \cite{ahlers2015,gimmon2010} \\
\factor{Investment Confidence} \\
      \group{Third Party Validation} \\
            Investment Record
                  & \cite{ahlers2015,beckwith2016,croce2016,hoenen2014,conti2013}
                  & - \\
            Investor Reputation
                  & \cite{an2015,werth2013,hsu2008}
                  & \cite{hoenen2014} \\
            Media Coverage
                  & \cite{beckwith2016}
                  & \cite{an2015} \\
      \group{Historical Performance} \\
            Financial Performance
                  & \cite{beckwith2016,baum2004}
                  & - \\
            Non-Financial Performance
                  & \cite{an2015,gimmon2010}
                  & \cite{hoenen2014} \\
      \group{Contextual Cues} \\
            Industry Performance
                  & \cite{shan2014,croce2016,gimmon2010}
                  & \cite{beckwith2016,conti2013} \\
            Broader Economy
                  & \cite{beckwith2016,croce2016,hoenen2014,conti2013,hsu2008}
                  & \cite{shan2014,ahlers2015} \\
            Local Economy
                  & \cite{shan2014,beckwith2016,croce2016,gimmon2010,hoenen2014}
                  & - \\
\bottomrule
\end{tabular}

    }
    \caption[Features relevant to \gls{vc} investment screening]{Features relevant to \gls{vc} investment investment. We review thirteen empirical studies that investigate drivers of \gls{vc} investment. For each study, we note whether included features have a significant effect on the \gls{vc} investment model.}
    \label{tab:litreview:features:summary}
\end{table}

\Gls{vc} investment decisions are made in an environment of informational asymmetry~\cite{ahlers2015}. Given this context, \gls{vc} investment decisions can be broken down into two main components: the underlying determinants of startup performance (which are difficult to observe), and signals that correlate with startup performance (which are easier to observe). We will review factors that underpin these two components in the following sections.

\subsection{Startup Potential}

Determinants of startup performance from the literature can be broadly categorised into three areas: human capital, social capital and structural capital~\cite{baum2004, ahlers2015}.

\begin{itemize}

\item Human Capital. Human capital is critical to early-stage startups that have limited resources and are changing constantly. The education background of founders~\cite{beckwith2016,gimmon2010}, and the past entrepreneurial experience of advisors~\cite{baum2004} and founders~\cite{gimmon2010} have been linked to startup performance, though some studies dispute this~\cite{shan2014,conti2013}.

\item Social Capital. Entrepreneurship is achieved through opportunity discovery and realisation, which occurs through the medium of social networks. Presence and engagement on Facebook and Twitter are predictive of startup performance~\cite{cheng2016,beckwith2016} and strategic alliances have also been found to predict \gls{vc} investment~\cite{baum2004}.

\item Structural Capital. Structural capital is the supportive intangible assets, infrastructure, and systems that enable a startup to function. Intellectual property and their proxy, patents, are a key component of structural capital for newly-formed startups. Patents and patent filings have been found to predict the survival and investment success of biotechnology startups~\cite{baum2004,hoenen2014} but there is less supportive evidence for non-biotechnology startups~\cite{gimmon2010,ahlers2015}.

\end{itemize}

\subsection{Investment Confidence}

A key challenge of the \gls{vc} investment process is informational asymmetry~\cite{ahlers2015}. To get an understanding of the underlying potential of a company, investors may look to other factors to corroborate the evidence like third party validation, historical performance, and contextual cues.

\begin{itemize}

\item Third-Party Validation. Founders are optimistic about their startups so it is reasonable for investors to cross-reference their signals with third-parties. Third-party validation from credible sources like other investors~\cite{ahlers2015,beckwith2016,croce2016,hoenen2014,conti2013}, the media~\cite{beckwith2016}, and the government has been shown to factor into \gls{vc} investors' decision-making processes.

\item Historical Performance. Unlike in other forms of finance, it is challenging to measure the performance of \gls{vc} candidates. Reporting is not standardised and profitability information is rarely available or too preliminary to be helpful. Simple performance metrics like survival time have been studied~\cite{an2015,gimmon2010} but are not helpful measures for \gls{vc} investment screening.

\item Contextual Cues. Startups do not exist in isolation but are rather a product of their context like any other business. Investors consider the performance of a startup's competitors~\cite{shan2014,croce2016,gimmon2010}, their local economy~\cite{beckwith2016,croce2016,gimmon2010} and the broader economy~\cite{croce2016,hoenen2014} when evaluating potential investment candidates.

\end{itemize}

\subsection{Feature Evaluation}

We collected evidence of features that influence startup performance and \gls{vc} investment decisions. We found that previous studies typically focused on factors in isolation and have rarely evaluated a comprehensive feature set. Without a standardised evaluation methodology, this has led to considerable disagreement between studies as to which factors are important. We believe a diverse range of features is critical to developing accurate models of startup performance and investment decisions. We recommend that \gls{vc} investment screening systems incorporate measures of the determinants of startup potential (human capital, social capital, and structural capital) and signals of investment confidence (third-party validation, historical performance, and contextual cues).

\section{Data Sources}

The identification of promising \gls{vc} investment opportunities is a complex and difficult task. There are many factors that can influence \gls{vc} investment decisions. Capturing the diversity of these factors is critical to developing accurate models. Appropriate selection of these data sources is important because different data sources provide insights into different actors, relationships and attributes. Ideally, tasks as complex as investment screening should involve data collection from multiple data sources.

Previous studies in this field have been limited by data sources restricted in sample size. Many studies have samples of fewer than 500 startups~\cite{ahlers2015, gimmon2010} or between 500 and 2,000 startups~\cite{hoenen2014, yu2015, an2015, werth2013, croce2016}. Few studies have used larger samples (more than 100,000 startups), usually derived from CrunchBase or AngelList~\cite{shan2014, cheng2016}. Sample size is more critical to model development than the sophistication of machine learning algorithms or feature selection~\cite{caruana2008}. Startup databases (e.g. CrunchBase) and social networks (e.g. Twitter) offer datasets larger than those used in many previous studies. We expect data collected from these sources will lead to the discovery of additional features and higher accuracy in \gls{vc} investment prediction.

In Table~\ref{fig:litreview:sources:summary}, we outline the characteristics of relevant data sources and how they could contribute to features indicated to be relevant to \gls{vc} investment decision-making. Furthermore, we describe desirable characteristics of data sources for \gls{vc} investment screening, review potentially relevant data sources, and ultimately determine which data sources are most likely to suit the characteristics of \gls{vc} investment screening.

\afterpage{
    \clearpage
        \begin{sidewaystable}[!htbp]
            \centering
            \scalebox{0.8}{
                
\newcommand{\type}[1]{\hspace{-6em}#1}
\newcommand{\factor}[1]{\hspace{-4em}#1}
\newcommand{\group}[1]{\hspace{-2em}#1}

\begin{tabular}{>{\hspace{6em}}lcccccc}
\toprule
\multicolumn{1}{l}{Properties} & \multicolumn{2}{c}{Startup Databases} & \multicolumn{2}{c}{Social Media} & \multicolumn{2}{c}{Other Sources}\\
\cmidrule(lr){2-3} \cmidrule(lr){4-5} \cmidrule(lr){6-7}
 & CrunchBase & AngelList & LinkedIn & Twitter & PatentsView & PrivCo \\
\midrule
\type{Features} \\
      \factor{Startup Potential} \\
            \group{Human Capital} \\
                  Founders' Capabilities %DONE
                        & \cmark & \cmark
                        & \cmark\cmark & \xmark
                        & \xmark & \xmark \\
                  NED Capabilities %DONE
                        & \cmark & \cmark
                        & \cmark\cmark & \xmark
                        & \xmark & \xmark \\
                  Staff Capabilities %DONE
                        & \cmark & \cmark
                        & \cmark\cmark & \xmark
                        & \xmark & \xmark \\
            \group{Social Capital} \\
                  Social Influence %DONE
                        & \cmark & \cmark\cmark
                        & \cmark\cmark & \cmark\cmark
                        & \xmark & \xmark \\
                  Strategic Alliances %DONE
                        & \cmark & \cmark
                        & \xmark & \xmark
                        & \cmark & \xmark \\
            \group{Structural Capital} \\
                  Patent Filings %DONE
                        & \xmark & \xmark
                        & \xmark & \xmark
                        & \cmark\cmark & \xmark \\
      \factor{Investment Confidence} \\
            \group{Third Party Validation} \\
                  Investment Record
                        & \cmark\cmark & \cmark\cmark
                        & \xmark & \xmark
                        & \xmark & \cmark \\
                  Investor Reputation
                        & \cmark & \cmark\cmark
                        & \cmark & \xmark
                        & \xmark & \xmark \\
                  Media Coverage
                        & \cmark\cmark & \cmark
                        & \xmark & \cmark
                        & \xmark & \xmark \\
                  Awards and Grants
                        & \cmark & \xmark
                        & \xmark & \xmark
                        & \xmark & \xmark \\
            \group{Historical Performance} \\
                  Financial Performance
                        & \xmark & \xmark
                        & \xmark & \xmark
                        & \xmark & \cmark\cmark \\
                  Non-Financial Performance
                        & \cmark\cmark & \cmark\cmark
                        & \cmark & \xmark
                        & \xmark & \cmark \\
            \group{Contextual Cues} \\
                  Competitor Performance
                        & \cmark & \cmark
                        & \xmark & \xmark
                        & \xmark & \xmark \\
                  Broader Economy
                        & \cmark & \cmark
                        & \xmark & \xmark
                        & \xmark & \xmark \\
                  Local Economy
                        & \cmark & \cmark
                        & \xmark & \xmark
                        & \xmark & \xmark \\
\type{Ease of Use} \\
      \factor{Cost Effective}
            & \cmark & \cmark\cmark
            & \cmark & \xmark
            & \cmark\cmark & \xmark \\
      \factor{Time Efficient}
            & \cmark\cmark & \cmark\cmark
            & \xmark & \cmark\cmark
            & \cmark\cmark & \xmark \\
      \factor{Accurate Data}
            & \cmark & \cmark
            & \cmark\cmark & \cmark\cmark
            & \cmark\cmark & \cmark\cmark \\
      \factor{Large Data Set}
            & \cmark\cmark & \cmark\cmark
            & \cmark\cmark & \cmark\cmark
            & \cmark\cmark & \cmark \\
\bottomrule
\end{tabular}

            }
            \caption[Data sources relevant to \gls{vc} investment screening]{Data sources relevant to \gls{vc} investment screening. We reviewed six data sources commonly used in entrepreneurship research for their suitability for \gls{vc} investment screening. We evaluated data sources on their ability to provide relevant features for our analyses and on their ease of use in data collection. We excluded offline sources from our analyses. Ratings are: \protect\xmark~=~poor, \protect\cmark~=~satisfactory, \protect\cmark\protect\cmark~=~good.}
            \label{fig:litreview:sources:summary}
        \end{sidewaystable}
    \clearpage
}

\subsection{Source Characteristics}

Entrepreneurship research is transforming with the availability of online data sources: databases, websites and social networks. Entrepreneurship studies have historically relied on surveys and interviews for data collection. Measures of human capital (e.g. founders' capabilities), strategic alliances, and financial performance are difficult to capture elsewhere. However, the trade-off for access to these features is that surveys and interviews are time-consuming and costly to implement. While online surveys address some of these issues, it is still difficult to motivate potential participants to contribute. Online data sources like startup databases and social networks are efficient because collecting data is a secondary function of users interacting with these sources. Researchers can also collect data from these sources automatically and at scale. For these reasons, we only consider online data sources for inclusion in this study, specifically crowd-sourced startup databases (e.g. CrunchBase, AngelList), social networks (e.g. Twitter, LinkedIn), government patent databases (e.g. PatentsView) and private company intelligence providers (e.g. PrivCo). We review the characteristics of each of these data sources commonly used in entrepreneurship research in Appendix~\ref{appendix:data_sources}.

\subsection{Source Evaluation}

Entrepreneurship and \gls{vc} investment research is primed to take advantage of new online data sources. We evaluated relevant data sources for their suitability to predicting startup investment. Startup databases CrunchBase and AngelList provide the most comprehensive set of features, including information on funding rounds, acquisitions, IPOs, employees, and investors. Both databases provide hundreds of thousands of company entries. There are small differences between the features recorded by each. CrunchBase has slightly more coverage but lacks AngelList's social network. At least one startup database should be used and either are satisfactory. Of the other data sources we reviewed, PatentsView is the most promising. PatentsView provides comprehensive patent information, though it could prove difficult matching identities to other sources. Other data sources are less promising because of access issues. LinkedIn cannot be easily collected now the API is deprecated. Twitter provides social network topology and basic profile information through its free API but does not provide access to historical tweets. Financial reports are too expensive for the purposes of this study.

\section{Classification Algorithms}

Predicting startup performance is a difficult problem for humans. However, in recent years, machine learning has been used successfully in other forms of finance and there may be scope to apply similar techniques to improve \gls{vc} investment screening. Machine learning is characterised by algorithms that improve their ability to reason about a given phenomenon given greater observation and/or interaction with said phenomenon. Mitchell (1997) provides a formal definition of machine learning in operational terms: ``A computer program is said to learn from experience E with respect to some class of tasks T and performance measure P if its performance at tasks in T, as measured by P, improves with experience E.''~\cite{mitchell1997}.

Machine learning algorithms can be classified based on the nature of the feedback available to them: supervised learning, where the algorithm is given example inputs and desired outputs; unsupervised learning, where no labels are provided and the algorithm must find structure in its input; and reinforcement learning, where the algorithm interacts with a dynamic environment to perform a certain goal. These algorithms can be further categorised by desired output: classification, supervised learning that divides inputs into two or more classes; regression, supervised learning that maps inputs to a continuous output space; and clustering, unsupervised learning that divides inputs into two or more classes.

We evaluated common machine learning algorithms with respect to their suitability for use in identifying startup investment potential. In Table~\ref{fig:litreview:algorithms:evaluation}, we rank these algorithms by cross-referencing their assumptions and properties with the task characteristics. In the following sections, we describe the characteristics of \gls{vc} investment screening, review common machine learning algorithms, and determine which algorithms are most likely to suit the characteristics of \gls{vc} investment screening.

\afterpage{
    \clearpage
        \begin{sidewaystable}[!htbp]
            \centering
            \setlength{\extrarowheight}{.5em}
            
\newcommand{\type}[1]{\hspace{-6em}#1}
\newcommand{\factor}[1]{\hspace{-4em}#1}
\newcommand{\group}[1]{\hspace{-2em}#1}

\begin{tabular}{>{\hspace{6em}}lcccccccccccccc}
\toprule
\multicolumn{1}{l}{Criteria} & \multicolumn{14}{c}{Machine Learning Algorithms} \\
\cmidrule(lr){2-15}
 & \multicolumn{2}{l}{NB} & \multicolumn{2}{l}{LR} & \multicolumn{2}{l}{KNN} & \multicolumn{2}{l}{DT} & \multicolumn{2}{l}{RF} & \multicolumn{2}{l}{SVM} & \multicolumn{2}{l}{ANN} \\
\midrule
\type{Data Set Properties}
            & \multicolumn{2}{c}{\textbf{2}}
            & \multicolumn{2}{c}{4}
            & \multicolumn{2}{c}{6}
            & \multicolumn{2}{c}{\textbf{2}}
            & \multicolumn{2}{c}{\textbf{1}}
            & \multicolumn{2}{c}{4}
            & \multicolumn{2}{c}{6}
      \\
\midrule
      \factor{Missing Values}
            & \cmark\cmark & \cite{kotsiantis2007}
            & \cmark & -
            & \xmark & \cite{kotsiantis2007}
            & \cmark\cmark & \cite{kotsiantis2007}
            & \cmark\cmark & \cite{strobl2009}
            & \cmark & \cite{kotsiantis2007}
            & \xmark & \cite{kotsiantis2007}
      \\
      \factor{Irrelevant Features}
            & \xmark & \cite{kotsiantis2007}
            & \xmark & \cite{kuhn2013}
            & \cmark & \cite{kotsiantis2007}
            & \cmark\cmark & \cite{kotsiantis2007}
            & \cmark\cmark & \cite{strobl2009}
            & \xmark & \cite{kotsiantis2007}
            & \xmark & \cite{kotsiantis2007}
      \\
      \factor{Imbalanced Classes}
            & \cmark\cmark & -
            & \cmark\cmark & -
            & \xmark & -
            & \xmark & \cite{kotsiantis2007}
            & \cmark & \cite{strobl2009}
            & \cmark\cmark & \cite{kotsiantis2007}
            & \cmark & \cite{kotsiantis2007}
      \\
\midrule
\type{Algorithm Properties}
            & \multicolumn{2}{c}{\textbf{2}}
            & \multicolumn{2}{c}{\textbf{1}}
            & \multicolumn{2}{c}{4}
            & \multicolumn{2}{c}{4}
            & \multicolumn{2}{c}{\textbf{2}}
            & \multicolumn{2}{c}{6}
            & \multicolumn{2}{c}{6}
      \\
\midrule
      \factor{Predictive Power}
            & \xmark & \cite{caruana2008}
            & \cmark & \cite{caruana2008}
            & \cmark & \cite{caruana2008}
            & \xmark & \cite{kotsiantis2007}
            & \cmark\cmark & \cite{caruana2008}
            & \cmark\cmark & \cite{caruana2008}
            & \xmark\cmark & \cite{caruana2008}
      \\
      \factor{Interpretability}
            & \cmark\cmark & \cite{kotsiantis2007}
            & \cmark\cmark & \cite{kuhn2013}
            & \xmark & \cite{kotsiantis2007}
            & \cmark\cmark & \cite{kotsiantis2007}
            & \cmark & \cite{kuhn2013}
            & \xmark & \cite{kotsiantis2007}
            & \xmark & \cite{kotsiantis2007}
      \\
      \factor{Processing Speed} %TODO
            & \cmark\cmark & \cite{kotsiantis2007}
            & \cmark\cmark & \cite{caruana2008}
            & \cmark\cmark & \cite{kotsiantis2007}
            & \cmark & \cite{kotsiantis2007}
            & \cmark & \cite{caruana2008}
            & \xmark  & \cite{kotsiantis2007}
            & \xmark  & \cite{kotsiantis2007}
      \\
\midrule
\type{Overall}
            & \multicolumn{2}{c}{\textbf{2}}
            & \multicolumn{2}{c}{\textbf{2}}
            & \multicolumn{2}{c}{6}
            & \multicolumn{2}{c}{4}
            & \multicolumn{2}{c}{\textbf{1}}
            & \multicolumn{2}{c}{5}
            & \multicolumn{2}{c}{7}
      \\
\bottomrule
\end{tabular}

            \caption[Classification algorithms relevant to \gls{vc} investment screening]{Evaluation of machine learning algorithms for use in \gls{vc} investment screening. We reviewed seven common supervised machine learning algorithms for their suitability in \gls{vc} investment screening. We evaluated algorithms for their robustness to the structure of the dataset and their appropriateness for the constraints of our implementation. We ranked the algorithms according to the sum of these measures (in each section and overall) and emphasised highly-ranked algorithms. Ratings are: \protect\xmark~=~poor, \protect\cmark~=~satisfactory, \protect\cmark\protect\cmark~=~good. Algorithms are: NB~=~Naive Bayes, LR~=~Logistic Regression, KNN~=~K-Nearest Neighbours, DT~=~Decision Trees, RF~=~Random Forests, SVM~=~Support Vector Machines, ANN~=~Artificial Neural Networks.}
            \label{fig:litreview:algorithms:evaluation}
        \end{sidewaystable}
    \clearpage
}

\subsection{Task Characteristics}

Machine learning tasks are diverse. \Gls{vc} investment screening is a task that suits supervised machine learning algorithms. We have access to historical labelled data, in the sense that we can use measures like whether a company has been acquired as a label of success. The key objective of machine learning algorithm selection is to find algorithms that make assumptions consistent with the structure of the problem (e.g. tolerance to missing values, mixed feature types, imbalanced classes) and suit the constraints of the desired solution (e.g. time available, incremental learning, interpretability). In the following sections, we outline the characteristics of supervised learning tasks relevant to \gls{vc} investment screening.

\subsubsection{Data Set Properties}

While datasets can be pre-processed to assist with their standardisation, some types of datasets are still better addressed by particular algorithms. Data set properties like missing data, irrelevant features, and imbalanced classes all have an effect on classification algorithms.

\begin{itemize}

\item Missing Values. Data sets often have missing values, where no data is stored for a feature of an observation. Missing data can occur because of non-response or because of errors in data collection or processing. Missing data has different effects depending on its distribution through the dataset. Public datasets, like startup databases and social networks, are typically sparse with missing entries despite their scale. Therefore, robustness to missing values is a desirable property of our algorithm.

\item Irrelevant Features. Despite efforts to only include features that have theoretical relevance, machine learning tasks often include irrelevant features. Irrelevant features have no underlying relationship with classification. Depending on how they are handled they may affect classification or slow the algorithm. We expect irrelevant and non-orthogonal features in datasets used in \gls{vc} investment screening because the features that predict startup performance have not been thoroughly evaluated in the literature. Therefore, robustness to irrelevant features is a desirable property of our algorithm.

\item Imbalanced Classes. Data sets are not usually restricted to containing equal proportions of different classes. Significantly imbalanced classes are problematic for some classifiers. In the worst case, a learning algorithm could simply classify every example as the majority class. Startup datasets are likely to be highly imbalanced because very few startups are successful. Therefore, robustness to imbalanced classes is a desirable property of our algorithm.

\end{itemize}

\subsubsection{Algorithm Properties}

The desired properties of machine learning algorithms are related to the business problems that are being addressed. Predictive power, interpretability and processing speed are all desirable characteristics but involve trade-offs and must be prioritised.

\begin{itemize}

\item Predictive Power. Predictive power is the ability of a machine learning algorithm to correctly classify new observations. If a model has no predictive power, the model is not representing the underlying process being studied. For this reason, predictive power is a desirable property of our algorithm. However, if multiple algorithms provide similar predictive power other selection criteria become significant.

\item Interpretability. Interpretability is the extent to which the reasoning of a model can be communicated to the end-user. There is a trade-off between model complexity and interpretability. Some models are a ``black box'' in the sense that data comes in and out but the model cannot be interpreted. For the purposes of investment screening, it is critical that \gls{vc} firms understand the logic being used by the system so they can trust its predictions. Therefore, interpretability is a desirable property.

\item Processing Speed. Finally, processing speed is another desirable property, especially when handling real-time data or when there is a need to run exploratory analyses on the fly. Generally \gls{vc} investment decisions are made over weeks and months, but there is still a need for the system to quickly process new information as it becomes available.

\end{itemize}

\subsection{Algorithm Characteristics}

Supervised machine learning are algorithms that reason about observations to produce general hypotheses that can be used to make predictions about future observations. Supervised machine learning algorithms are diverse, from symbolic (Decision Trees, Random Forests) to statistical (Logistic Regression, Naive Bayes, Support Vector Machines), instance-based (K-Nearest Neighbours), and perceptron-based (Artificial Neural Networks). In Appendix~\ref{appendix:classification_algorithms}, we describe each candidate learning algorithm, critique their advantages and disadvantages, and present evidence of their effectiveness in applications relevant to \gls{vc} investment.

\subsection{Algorithm Evaluation}

We evaluated supervised learning algorithms for their suitability for use in \gls{vc} investment screening systems. While our evaluation gives directionality of fit, we hesitate to discard algorithms based on our literature review. Algorithm selection is complex and preliminary testing in the following chapter will provide clarity as to which algorithms should be used. In addition, larger training sets and good feature design tend to outweigh algorithm selection~\cite{caruana2008}. With those concessions aside, our review suggests we should expect Random Forests, Support Vector Machines and Artificial Neural Networks to produce highest classification accuracies. An ensemble of these algorithms may improve accuracy further, though at the cost of computational speed and interpretability. We may expect Random Forests to outperform the other two algorithms because of robustness to missing values and irrelevant features and native handling of discrete and categorical data. However, Random Forests are not highly interpretable so Decision Trees and Logistic Regression may be preferable for exploratory analysis of the dataset.

\section{Research Gap}

As the cost of starting businesses decreases, \gls{vc} firms are faced with an overwhelming number of investment candidates to assess and evaluate. The \gls{vc} industry requires better systems and processes to efficiently manage labour-intensive tasks like investment screening. Attempts to address this problem have three common limitations: small sample size~\cite{ahlers2015, gimmon2010, hoenen2014, yu2015, an2015, werth2013, croce2016}, a focus on early-stage investment~\cite{beckwith2016, ahlers2015, cheng2016, yuan2016, croce2016, stone2014}, and a basic feature set~\cite{ahlers2015, an2015, cheng2016, croce2016, werth2013, gimmon2010}. In addition, there is little evidence that previous research has been translated into systems that are able to assist investors directly. We conducted a literature review to determine whether there was scope to address these limitations and produce a system that assists \gls{vc} firms in screening investment candidates.

First, we reviewed the business context and developed criteria to guide the development and evaluation of our system: practicality, robustness and versatility. Second, we reviewed studies that have developed models of startup potential and realised few individual studies have evaluated a comprehensive and diverse feature set. Third, we assessed potential data sources and found preliminary evidence that suggests that the startup databases CrunchBase and AngelList are promising. Finally, we reviewed supervised machine learning techniques as applied to startup investment and other areas of finance. Our analyses suggested that we should expect Random Forests, Support Vector Machines and Artificial Neural Networks to be most suitable for our system.

This literature review provided evidence to suggest that it is possible to address previous limitations in this domain and produce an improved \gls{vc} investment screening system that is practical, robust, and versatile. In the next chapter, we outline the process by which we developed that system.

\ifcsdef{mainfile}{}{
    \appendix
    \subfile{../litreview/appendices}
    \printbibliography
}
\end{document}

