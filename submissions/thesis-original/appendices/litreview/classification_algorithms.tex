\chapter{Classification Algorithms}
\label{appendix:classification_algorithms}

\section{Naive Bayes}

Naive Bayes is a simple generative learning algorithm. It is a Bayesian Network that models features by generating a directed acyclic graph, with the strong (naive) assumption that all features are independent. While this assumption is generally not true, it simplifies estimation which makes Naive Bayes more computationally efficient than other learning algorithms. Naive Bayes can be a good choice for data sets with high dimensionality and sparsity as it estimates features independently. Naive Bayes sometimes outperforms more complex machine learning algorithms because it is reasonably robust to violations of feature independence \cite{kotsiantis2007}. However, Naive Bayes is known to be a poor estimator of class probabilities, especially with highly correlated features \cite{niculescu2005}. Naive Bayes was used alongside Logistic Regression, Decision Trees and Support Vector Machines to predict success in equity crowdfunding campaigns on the AngelList data set \cite{beckwith2016}. None of these models performed well. The algorithm that best predicts startup investment was Naive Bayes with a Precision of .41 and Recall of .19, which means only 19\% of funded startups were classified correctly by the model. The author suggests the poor performance of their algorithms is caused by features not captured in their data set relating to Intellectual Capital, Third Party Validation and Historical Performance. These features will be included in this study.

\section{Logistic Regression}

Regression is a class of statistical methods that investigates the relationship between a dependent variable and a set of independent variables. Logistic regression is regression where the dependent variable is discrete. Like linear regression, logistic regression optimises an equation that multiplies each input by a coefficient, sums them up, and adds a constant. However, before this optimisation takes place the dependent variable is transformed by the log of the odds ratio for each observation, creating a real continuous dependent variable on a logistic distribution. A strength of Logistic Regression is that it is trivial to adjust classification thresholds depending on the problem (e.g. in spam detection \cite{friedman2001}, where specificity is desirable). It is also simple to update a Logistic Regression model using online gradient descent, when additional training data needs to be quickly incorporated into the model (incremental learning). Logistic Regression tends to underperform against complex algorithms like Random Forest, Support Vector Machines and Artificial Neural Networks in higher dimensions \cite{caruana2008}. This underperformance is observed when Logistic Regression is applied to startup investment prediction tasks \cite{beckwith2016, bhat2011}. However, weaker predictive performance has not prevented Logistic Regression from being commonly used. Its simplicity and ease-of-use means it is often used without justification or evaluation \cite{gimmon2010}.

\section{K-Nearest Neighbours}

K-Nearest Neighbours is a common lazy learning algorithm. Lazy learning algorithms do not produce explicit general models, but compare new instances with instances from training stored in memory. K-Nearest Neighbours is based on the principle that the instances within a data set will exist near other instances that have similar characteristics. K-Nearest Neighbours models depend on how the user defines distance between samples; Euclidean distance is a commonly used metric. K-Nearest Neighbour models are stable compared to other learning algorithms and suited to online learning because they can add a new instance or remove an old instance without re-calculating \cite{kotsiantis2007}. A shortcoming of K-Nearest Neighbour models is that they can be sensitive to the local structure of the data and they also have large in-memory storage requirements. K-Nearest Neighbours was compared to Artificial Neural Networks to predict firm bankruptcy \cite{ahn2008}. K-Nearest Neighbours is attractive in bankruptcy prediction because it can be updated in real-time. By optimising feature weighting and instance selection, the authors improved the K-Nearest Neighbours algorithm to the extent that it outperformed the Artificial Neural Networks.

\section{Decision Trees}

Decision Trees use recursive partitioning algorithms to classify instances. Each node in a Decision Tree represents a feature in an instance to be classified, and each branch represents a value that the node can assume. Methods for finding the features that best divide the training data include Information Gain and Gini Index \cite{kotsiantis2007}. Decision Trees are close to an ``off-the-shelf'' learning algorithm. They require little pre-processing and tuning, are interpretable to laypeople, are quick, handle feature interactions and are non-parametric. However, Decision Trees are prone to overfitting and have poor predictive power \cite{caruana2006}. These shortcomings are addressed with pruning mechanisms and ensemble methods like Random Forests, respectively. Decision Trees were compared with Naive Bayes and Support Vector Machines to predict investor-startup funding pairs using CrunchBase social network data \cite{liang2016}. Decision Trees had the highest accuracy and are desirable because their reasoning is easily communicated to startups.

\section{Random Forests}

Random Forests are an ensemble learning technique that constructs multiple Decision Trees from bootstrapped samples of the training data, using random feature selection \cite{breiman2001}. Prediction is made by aggregating the predictions of the ensemble. The rationale is that while each Decision Tree in a Random Forest may be biased, when aggregated they produce a model robust against over-fitting.  Random Forests exhibit a performance improvement over a single Decision Tree classifier and are among the most accurate learning algorithms \cite{caruana2006}.  However, Random Forests are more complex than Decision Trees, taking longer to create predictions and producing less interpretable output. Random Forests were used to predict private company exits using quantitative data from ThomsonOne \cite{bhat2011}. Random Forests outperformed Logistic Regression, Support Vector Machines and Artificial Neural Networks. This may be because the data set was highly sparse, and Random Forests are known to perform well on sparse data sets \cite{breiman2001}.

\section{Support Vector Machines}

Support Vector Machines are a family of classifiers that seek to produce a hyperplane that gives the largest minimum distance (margin) between classes. The key to the effectiveness of Support Vector Machines are kernel functions. Kernel functions transform the training data to a high-dimensional space to improve its resemblance to a linearly separable set of data. Support Vector Machines are attractive for many reasons. They have high predictive power \cite{caruana2006}, theoretical limitations on overfitting, and with an appropriate kernel they work well even when data is not linearly separable in the base feature space. Support Vector Machines are computationally intensive and complicated to tune effectively (compared to Random Forests, for example). Support Vector Machines were compared with back propagated Artificial Neural Networks in predicting the bankruptcy of firms using data provided by Korea Credit Guarantee Fund \cite{shin2005}. Support Vector Machines outperformed Artificial Neural Networks, possibly because of the small data set.

\section{Artificial Neural Networks}

Artificial Neural Networks are a computational approach based on a network of neural units (neurons) that loosely models the way the brain solves problems. An Artificial Neural Network is broadly defined by three parameters: the interconnection pattern between the different layers of neurons, the learning process for updating the weights of the interconnections, and the activation function that converts a neuron's weighted input to its output activation. A supervised learning process typically involves gradient descent with back-propagation \cite{rumelhart1988}. Gradient descent is an optimisation algorithm that updates the weights of the interconnections between the neurons with respect to the derivative of the cost function (the weighted difference between the desired output and the current output). Back-propagation is the technique used to determine what the gradient of the cost function is for the given weights, using the chain rule. Artificial Neural networks tend to be highly accurate but are slow to train and require significantly more training data than other machine learning algorithms. Artificial Neural Networks are also a black box model so it is difficult to reason about their output in a way that can be effectively communicated. Artificial Neural Networks are rarely applied to startup investment or performance prediction because research in this area typically uses small and low-dimensional data sets. As one author puts it ``More complex classification algorithms —- artificial neural networks, Restricted Bolzmann machines, for instance —- could be tried on the data set, but marginal improvements would likely result.'' \cite{beckwith2016}. However, this study will address these issues so Artificial Neural Networks may be more competitive.
