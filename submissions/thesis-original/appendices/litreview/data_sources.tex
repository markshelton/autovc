\chapter{Data Sources}
\label{appendix:data_sources}

\section{Databases} %Example data entries?

Databases play a critical role in understanding the startup ecosystem, aggregating information about startups, investors, media and trends. Most startup databases are closed systems that require commercial licenses (e.g. CB Insights, ThomsonOne, Mattermark). CrunchBase and AngelList are two crowd-sourced and free-to-use alternatives. AngelList’s primary function is as an equity crowdfunding platform but it has a data-sharing agreement with CrunchBase which results in significant overlap between the two sources. CrunchBase and AngelList provide free Application Program Interfaces (API) for academic use. Crawlers can be developed to traverse these APIs and collect data systematically. The advantages of crawlers are that they can selectively collect data from nodes with specific attributes, collect random samples, or traverse the data source indefinitely, updating entries as new data becomes available. CrunchBase also provides pre-formatted database snapshots which allows easier access to the data set. The crowd-sourced nature of CrunchBase and AngelList has advantages and limitations . The key advantages are that access to the databases is free and the dataset is relatively comprehensive. The limitations are that both CrunchBase and AngelList have relatively sparse profiles (i.e. limited depth), particularly for unpopular startups. Both CrunchBase and AngelList also have error-checking provisions (including machine reviews and social authentication) to prevent and remediate inaccurate entries but there is still a greater chance for error. Comparing CrunchBase and AngelList, CrunchBase tends to have more comprehensive records of funding rounds \cite{cheng2016} and media coverage but AngelList also has a social network element where users can `follow’ each other - in a similar way to Twitter.

\section{Social Networks}

Social networks provide an interesting perspective into the process of opportunity discovery and capitalisation that characterises entrepreneurship. Two social networks studied in detail in entrepreneurship research are LinkedIn and Twitter. LinkedIn is a massive professional social network often used in studies of entrepreneurship for measures of employment, education and weak social links. These measures are difficult to collect elsewhere. In addition, LinkedIn can provide a measure of the professional influence of founders and investors. Unfortunately, as of May 2015, the LinkedIn API no longer allows access to authenticated users' connection data or company data \cite{trachtenberg2015}, making it difficult to use for social network analyses. Twitter is a massive social networking and micro-blogging service which is studied in entrepreneurship research because it is used by founders, investors, and customers to quickly communicate and broadcast. Twitter is a directed network where users can follow other users without gaining their permission to do so. Twitter's public API provides access to social network topological features (e.g. who follows who) and basic profile information (e.g. user-provided descriptions). However, Twitter's API only provides Tweets published within the last 7 days and access to historical Twitter data requires a commercial license \cite{puschmann2013}.

\section{Other Sources}

While startup databases and social networks provide a variety of information on startups, there are two important areas that they do not cover: patent filings and financial performance. Startups often file patents to apply for a legal right to exclude others from using their inventions. In 2015, the US Patents Office (USPTO) launched PatentsView, a free public API to allow programmatic access to their database. PatentsView holds over 12 million patent filings from 1976 onwards \cite{schultz2016}. The database provides comprehensive information on patents, their inventors, their organisations, and locations. It may be difficult to match identities across PatentsView to other data sources because registered company names (as in PatentsView) are not always the same as trading names (as elsewhere). Finding other information on startups, like financial information, is difficult. Unlike public companies, private companies are not required to file with the United States Securities and Exchange Commission (or international equivalent). Proprietary databases provide some data on private companies but commercial licenses are prohibitively expensive and have poor coverage of early-stage companies. PrivCo is one of few commercial data sources for private company business and financial intelligence. PrivCo focuses its coverage on US private companies with at least \$50-100 million in annual revenues but also has some coverage on smaller but high-value private companies (like startups) \cite{artemchik2015}.
