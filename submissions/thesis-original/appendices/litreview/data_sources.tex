\chapter{Data Sources}
\label{appendix:data_sources}

This appendix provides an extended overview of potential online data sources relevant to developing a \gls{vc} investment screening system, including startup databases, social networks, and other sources.

\section{Startup Databases}

Databases play a critical role in understanding the startup ecosystem, aggregating information about startups, investors, media and trends. Most startup databases are closed systems that require commercial licenses (e.g. CB Insights, ThomsonOne, Mattermark). CrunchBase and AngelList are two crowd-sourced and free-to-use alternatives.

\subsection{CrunchBase}

CrunchBase is an open online database of information about startups, investors, media coverage and trends, focusing on high-tech industry in the United States. It relies on its active online community to contribute to and edit most of its pages. However, this results in unpopular startups having relatively sparse profiles. CrunchBase has three provisions to prevent and remediate inaccurate crowd-sourced entries. First, users authenticate their accounts with a social media account which allows CrunchBase to verify a user's identity. Second, every change goes through a machine review, which flags significant or questionable updates. Third, established startups have their editing privileges locked and updates require manual verification.

\subsection{AngelList}

AngelList combines the functionality of an equity crowdfunding platform, a social networking site and an online startup database. As an equity crowdfunding platform, users create profiles for their startups on AngelList, and use the platform to attract investment. Investors use the platform to identify investment opportunities and can invest directly through AngelList, often alongside other investors in investment syndicates. AngelList is also an online startup database. It has a data-sharing agreement with CrunchBase which results in significant overlap between the two sources, though CrunchBase tends to have more comprehensive records of funding rounds \cite{cheng2016}. AngelList tracks ``startup roles'' (e.g. founders, investors, employees) with a creation-time, start-time and end-time. This means that, unlike CrunchBase, AngelList's networks can be re-created through time, which is useful for longitudinal studies.

\subsection{Comparison}

AngelList’s primary function is as an equity crowdfunding platform but it has a data-sharing agreement with CrunchBase which results in significant overlap between the two sources. CrunchBase tends to have more comprehensive records of funding rounds \cite{cheng2016} and media coverage but AngelList also has a social network element where users can `follow’ each other -- in a similar way to Twitter. The crowd-sourced nature of CrunchBase and AngelList has advantages and limitations. The key advantages are that access to the databases is free and the dataset is relatively comprehensive. The limitations are that both CrunchBase and AngelList have relatively sparse profiles (i.e. limited depth), particularly for unpopular startups.

\section{Social Networks}

Social networks provide an interesting perspective into the process of opportunity discovery and capitalisation that characterises entrepreneurship. Two social networks studied in detail in entrepreneurship research are LinkedIn and Twitter.

\subsection{LinkedIn}

LinkedIn is a massive professional social network often used in studies of entrepreneurship for measures of employment, education and weak social links. These measures are difficult to collect elsewhere. In addition, LinkedIn can provide a measure of the professional influence of founders and investors. Unfortunately, as of May 2015, the LinkedIn API no longer allows access to authenticated users' connection data or company data, making it difficult to use for social network analyses.

\subsection{Twitter}

Twitter is a massive social networking and micro-blogging service which is studied in entrepreneurship research because it is used by founders, investors, and customers to quickly communicate and broadcast. Twitter is a directed network where users can follow other users without gaining their permission to do so. Twitter's public API provides access to social network topological features (e.g. who follows who) and basic profile information (e.g. user-provided descriptions). However, Twitter's API only provides Tweets published within the last 7 days and access to historical Twitter data requires a commercial license.

\section{Other Sources}

While startup databases and social networks provide a variety of information on startups, there are two important areas that they do not cover: patent filings and financial performance.

\subsection{PatentsView}

Startups often file patents to apply for a legal right to exclude others from using their inventions. In 2015, the US Patents Office (USPTO) launched PatentsView, a free public API to allow programmatic access to their database. PatentsView holds over 12 million patent filings from 1976 onwards. The database provides comprehensive information on patents, their inventors, their organisations, and locations. It may be difficult to match identities across PatentsView to other data sources because registered company names (as in PatentsView) are not always the same as trading names (as elsewhere).

\subsection{PrivCo}

Finding other information on startups, like financial information, is difficult. Unlike public companies, private companies are not required to file with the United States Securities and Exchange Commission (or international equivalent). Proprietary databases provide some data on private companies but commercial licenses are prohibitively expensive and have poor coverage of early-stage companies. PrivCo is one of few commercial data sources for private company business and financial intelligence. PrivCo focuses its coverage on US private companies with at least \$50-100 million in annual revenues but also has some coverage on smaller but high-value private companies (like startups).
