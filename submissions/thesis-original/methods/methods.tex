\documentclass[../thesis/thesis.tex]{subfiles}
\begin{document}

\chapter{Design}
\label{chap:design}

In this chapter, we explain the methodology used to fill the research gap identified in Chapter~2, thereby producing a system that identifies startup companies likely to receive additional funding or a liquidity event (exit) in a given forecast window. Figure [FIGURE] depicts the system architecture, which we describe in detail in the following sections.

\begin{enumerate}

\item Data Collection: We collected data from the CrunchBase online database, with supplementation from PatentsView (US Patents Office). We collected two CSV dumps from CrunchBase in September 2016 and April 2017, for training and (final) testing respectively. We imported the CSV dumps into a relational database and performed aggregation queries to create features suitable for classification. We performed screening based on company developmental stage and age to ensure only relevant companies were included in the dataset.

\item Classification Pipeline: We developed a processing pipeline using the popular Python-based machine learning library Scikit-learn \cite{pedregosa2011}. The pipeline includes imputation, transformation and scaling, extraction and classification. We optimised these steps using cross-validated randomized hyperparameter search. Key dataset issues identified included extreme sparsity, long-tailed skewed distributions and non-orthogonal features. Classification evaluation revealed Random Forests and Logistic Regression are best suited to the problem.

\end{enumerate}

\section{Data Collection}

In the previous chapter, we reviewed the literature concerning data sources for entrepreneurship and VC investment. We concluded the most promising primary data sources for this project are CrunchBase and AngelList, for their size, comprehensiveness and ease of access. We suggested PatentsView (the online database of the US Patent Office) could be a useful secondary data source for structural capital features. In the following sections, we discuss how we collected data from CrunchBase and PatentsView, converted the relational databases into a format suitable for machine learning, performed preliminary screening to ensure we only included relevant companies, and finally, we provide an overview of the primary dataset.

\subsection{CrunchBase}

CrunchBase is an online, crowd-sourced repository of startup companies, individuals and investors with a focus on US high-tech sectors. CrunchBase is freely accessible for browsing but requires licenses to filter the dataset, use the API, and download Microsoft Excel and CSV-formatted dumps. For the purposes of this project, we were granted an Academic License. CrunchBase provides database access in a few formats that offer trade-offs in terms of accessibility and comprehensiveness. We intended to use CrunchBase’s API because it provides the most comprehensive access to their database. We developed a collector that downloaded a daily list of updated API endpoints from CrunchBase and queried nodes it needed to update. CrunchBase’s API provides JSON-formatted responses which the program recursively parsed and stored into a relational database. However, due to the time constraints of this research project, we abandoned this data collection method. CrunchBase also provides CSV-formatted dumps of their key endpoints (e.g. organizations, people, funding\_rounds). We downloaded two CSV-formatted dumps from Crunchbase on 09-Sep-2016 and 04-Apr-2017 which we loaded into relational databases (see Appendix D for the full database schema).

\subsection{PatentsView}

In 2015, the US Patents Office (USPTO) launched PatentsView, a free public API to allow programmatic access to their database. PatentsView holds over 12 million patent filings from 1976 onwards \cite{schultz2016}. The database provides comprehensive information on patents, their inventors, their organisations, and locations. We collected the patent filing records of each company in the primary database, focusing on information relating to dates, citations, and patent types. We matched the data sources on standardised company names (removing common suffixes, punctuation etc.). Although this approximate matching introduces error, the volume of companies in the database is too high to be matched manually and there are no other identifying records. We stored the PatentsView data in a relation in the primary database.

\subsection{Dataset Preparation}

To prepare the dataset for machine learning, we first flattened the relational database into a single file using SQL aggregation queries. We aggregated each relevant relation in turn, grouping by Company ID and then combined each aggregated table using left outer joins. Following this process, we used Python to convert tuples (e.g. Round Type and Round Date), lists (e.g. Round Types) and categorical data (e.g. HQ Country) into dummy variables. We then performed preliminary screening on the primary dataset (N = 425,934) to ensure it only included relevant companies. We were interested in removing traditional, non-startup businesses from the dataset (e.g. consulting firms, companies that will not take VC funding etc.). To do this, we explored two factors for each company: developmental stage and age.

First we grouped the dataset by startup developmental stage, as depicted in Figure 3.2. The lifecycle of a startup can be roughly divided into stages that reflect changes in their functions and objectives. Formally, these stages are associated with external funding milestones, but we also expect them to correlate with company age. We decided to further separate companies that had not raised funding into two categories (New and Other). We assume that companies that have not raised funding fall into two groups - those that intend to raise funding but have not had time to yet, and those that have chosen not to pursue funding and are unlikely to do so. We separated these groups using a Gaussian Mixture model based on their bimodal age distribution. Accordingly, companies grouped into the Other category included older companies (median: 19.0 years) and companies that had raised debt funding or other alternative funding. On this basis, we decided to exclude the Other group from further analyses (N = 61,265, 14.4\%). As we are only interested in companies that could theoretically seek investment, we also excluded Closed, Acquired and IPO groups from further analyses (N = 35,973, 8.4\%).

Overall, these preliminary screening steps reduced the dataset from 425,934 companies to 325,722 companies, a reduction of 23.5\%. Table 3.1 presents the descriptive statistics for the dataset. The dataset is heavily skewed towards New companies (i.e. companies that were recently founded and have not raised any type of funding yet, 84.8\%). The interquartile ranges imply significant variability in all measures. In addition, the sparsity of the dataset is high, particularly for New companies.

CrunchBase’s approach to industry classification is simplistic compared to other classification schemes (e.g., USSIC, NAICS, VentureSource) which generally have an industry hierarchy with tiers for broad industry sectors and sub-sectors providing further granularity. As a result, CrunchBase class labels include over represented and vague classes (e.g., “Software”, “Internet Services”) which could describe the majority of companies included in the database. In fact, “Software” and “Internet Services” account for 16.4\% and 13.4\% of all companies in the dataset respectively (see Figure 3.4). Despite these vague class labels, it is clear the dataset skews towards high technology startups, as opposed to biomedical, agricultural, or other technologies.

\section{Classification Pipeline}

We developed a classification pipeline using the popular Python-based machine learning library Scikit-learn \cite{pedregosa2011}. The classification pipeline construct allows us to easily search across hyperparameters at each step in the pipeline. The following sections discuss preliminary exploration of the dataset, empirical testing of each hyperparameter decision, and the selection of primary classifiers for the following experiments.

\subsection{Imputation}

First, we explored missing data in the dataset. We expected the dataset to be highly sparse because it primarily came from CrunchBase, a crowd-sourced database. As profiles are entered into CrunchBase piece-meal, it is not clear at face-value whether data (e.g. records of funding rounds) is missing or didn’t occur. Figure 3.5 and Figure 3.6 display the distribution of missing data in the dataset, with respect to each company and each feature. The multi-modal peaks of Figure 3.5 suggest that missing data across certain groups of features may be correlated with each other (e.g. all features derived from funding rounds).  Figure 3.6 shows that many features have no missing observations - these are probably core profile features and features where missing data is represented as zero-valued.

After reviewing the distribution of missing data, we decided to perform further investigation into imputation methods. Common imputation strategies include replacing missing values with the mean, median or mode of each feature. Figure 3.7 shows the distribution of mean, median and modes for each feature in the dataset. For the majority of features, all three measures of central tendency are equal to zero. This resolves the issue of distinguishing missing data from negative observations because, following imputation, all of these data points will map to zero. Figure 3.8 shows the receiver-operating characteristics of the different imputation strategies. As expected, all three imputation strategies produce similar results (within the margin of error).

\subsection{Transformation \& Scaling}

Next, we explored the distributions of features. While the classification algorithms we identified in the previous chapter are relatively robust to violations of normality, it may be beneficial to transform the data if the feature distributions are extreme. Figure 3.9 shows one of the key features, Total Funding Raised, under a number of different transformations. As we identified in the last section, the dataset is highly sparse and that is apparent in these distributions - the majority of observations are zero-valued. The distribution of Total Funding Raised is also highly skewed. The log transformation reduces this skew (a normal distribution of non-zero values is apparent) and square root transformation also reduces this skew (to a lesser extent). It is reasonable to expect both of these transformations to improve the classification accuracy. Figure 3.10 shows the Receiver Operating Characteristics of these different transformation functions. Both functions provide a small performance improvement, with the square root function narrowly best.

Standardisation of datasets is a common requirement for many feature extraction methods and machine learning estimators. Sci-kit learn provides three primary scaling functions: StandardScaler, RobustScaler and MinMaxScaler. RobustScaler is intended to alleviate the effect of outliers while MinMaxScaler is intended to preserve zero entries in sparse data - both of these are relevant properties for the dataset. Figure 3.11 shows the receiver-operating characteristics of the different scaling functions. MinMaxScaler and RobustScaler actually underperform the null condition while StandardScaler only performs on par with the null condition. This is unexpected but may be caused by the previously applied transformations.

\subsection{Extraction}

Feature extraction reduces high-dimensional data into lower-dimensional data in such a way that maximises the variance of the data. The most common approach to dimensionality reduction is Principal Components Analysis (PCA), which constructs orthogonal eigenvectors (components). The magnitude of each eigenvector (its eigenvalue) is displayed in Figure 3.12. The majority of explained variance is captured in the first 10 components, and the Eigenvalues drop below 1 by 100 components - this suggests that these are reasonable values for further hyperparameter search. Figure 3.13 shows the Receiver Operating Characteristics for different numbers of extracted components. All curves produce similar classification results (within margin of error) which implies that we should extract between 1 - 20 components (Group 0) because it will provide us with more efficient computation.

While PCA is efficient at reducing features, the resultant components are not interpretable. Similarly, individual analysis of 400+ features is difficult to interpret. A compromise is to group the features using the conceptual framework we developed earlier from the literature review. The grouping approach applied weights to each individual feature that optimised the inter-correlations within each group. Given the highly skewed features, we use Spearman correlation which is robust to skewness because it is based on ranking. Figure 3.14 displays the inter-correlations between each factor from the proposed conceptual framework. As we would expect, investors and funding features are highly correlated. Interestingly, founder features are positively correlated with all other features except for advisor features which are negatively correlated with all other feature groups.

\subsection{Classification Algorithms}

The literature review we performed in the previous chapter revealed seven common supervised classification algorithms potentially suitable for this problem area. Our review suggested that Random Forests were most likely to provide a successful trade-off between predictive power, interpretability and time taken. We empirically tested each of these classifiers and compared their performance against a range of metrics, as displayed in Table 3.3. We made our comparisons based on maximum recorded scores rather than measures of central tendency to ensure we didn’t penalise algorithms that had unfavourable hyperparameter search spaces.

\subsection{Conclusion}

In the previous sections, we outlined the development and optimisation of a classification pipeline that includes imputation, transformation and scaling, extraction and classification. We evaluated the predictive power, interpretability, and time efficiency of the most promising candidate pipelines. The best candidate pipeline is depicted in Table 3.4. We adopted this pipeline configuration for the following experiments.

\ifcsdef{mainfile}{}{\printbibliography}
\end{document}
